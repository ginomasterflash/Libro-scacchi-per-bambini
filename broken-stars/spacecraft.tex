% =================
% Spacecraft
% =================

\section{Spacecraft}

\begin{multicols}{2}

\subsection{Spacecraft Creation}

\begin{enumerate}

  \item \textbf{Hull}: Select a Hull type. This will give your spacecraft some defaults and base values such as crew size, storage space and available power.
  
  \item \textbf{Rank}: All spacecraft have a rank to help designate it's quality. These ranks are Novice, Seasoned, Veteran, Heroic and Legendary. New crews will always start with a Novice spacecraft, although the GM may allow for better craft. Each rank above Novice gains your spacecraft 4 advancement points.

  \item \textbf{Attributes}: You start with a d4 in each attribute and have 5 points with which to raise them. Raising an attribute by one die type costs 1 point (but you cannot raise an attribute above a d12).

  \begin{redtable}{\linewidth}{ @{} L{1} @{}}
    \textbf{Armour}\\
    Represents Hull integrity and ability to take damage\\
    \textbf{Engines}\\
    Determines acceleration and maximum impulse velocity\\
    \textbf{Power}\\
    How much power is available to power all of the systems.\\
    \textbf{Bulk}\\
    The size of the ship, relative to others in it's class\\
    \textbf{Systems}\\
    Determines the effectiveness of spacecraft systems\\
  \end{redtable}

  \item \textbf{Systems}: You have 10 points for systems. Systems allow the spacecraft to automatically handle common craft tasks. Each die type in a system costs 1 point up to your System attribute (going over your System attribute will cost 2 points per die type). Each system has a linked Skill; when a character uses the system, they roll their linked Skill to see if they can confer a bonus to the system roll (see Cooperation rules). If you do not have a die type in a system, then you have to manually operate that system.
  \begin{redtable}{\linewidth}{ @{} L{.5} @{} L{.5} @{} }
    \textbf{System} & \textbf{Linked Skill}\\
    Auto-pilot  & Piloting\\
    ECM         & Operations\\
    Navigation  & Astrogation\\
    Operations  & Smarts\\
    Repair      & Repair\\
    Sensors     & Notice\\
    Weapons     & Shooting\\
  \end{redtable}

  \item \textbf{Derived Traits}: Calculate your derived traits according using your attributes and skills.

  \begin{redtable}{\linewidth}{ @{} L{1} @{}}
    \textbf{Cargo}\\
    Calculated as: \textbf{(Excess Storage * 500 kg}. How much cargo you can carry.\\
    \textbf{Evade}\\
    Calculated as: \textbf{(Autopilot / 2) +2}. Target number for attackers when autopilot is on. If manually operated, make an opposed roll using the Piloting skill.\\
    \textbf{Toughness}\\
    Calculated as: \textbf{(Armour / 2) + 2}. Add any additional armour upgrades on top of toughness.\\
  \end{redtable}

  \item \textbf{Hindrances}: You can choose to gain additional character creation points by taking up to \textbf{one} Major Hindrance (2 points) and up to \textbf{two} Minor Hindrances (1 point each).

  \begin{itemize}
      \item For 2 points you can gain another attribute point or choose an \textbf{Edge}
      \item For 1 point you can gain another systems point
  \end{itemize}

  \item \textbf{Edges}: Edges are what sets your ship apart from everyone else.

  \item \textbf{Gear}: New crews are given a budget of up to 350k to equip their spacecraft. You can ignore this budget if you are purchasing with your own hard-earned credits.

  \item \textbf{Background Detail}: Fill in any other details of your spacecraft's background such as manufacturer, model, past crews and any other detail you like.
  
  \item \textbf{Calculate Value}: Calculate your spacecraft's final value. This is the purchase price for the spacecraft, and is usually the size of the debt owed by new crews.

\end{enumerate}

\subsection{Spacecraft Rules}

\begin{itemize}
  \item \textbf{Shaken} Spacecraft can be shaken the same as characters do. If a vessel is shaken it cannot use any of its systems or weapons. The spacecraft counts as an unstable platform and all player actions suffer a -2 penalty as long as it remains shaken. A character must make a "Power" roll to remove the Shaken status and return all systems back to normal operation.
  \item \textbf{Wounds} Spacecraft have wounds just like a character does, and it incurs the same penalties. It is incapacitated if a ship suffers more wounds than it's limit. Immediately make a "Power" roll modified by the ship's wound penalties.
  \begin{itemize}
    \item On a 1 or less then the vessel is destroyed. The crew has an amount of rounds equal to the Armour die type to evacuate before it explodes and takes everyone with it. 
    \item On a failure you roll 2d6 on the Damage Table. The damage is permanent and can only be removed through repair in space dock. Some damage is so grave it can never be properly repaired.
    \item On a success the damage is only temporary and will be removed as soon as all wounds are repaired. If you roll a raise the damage is only temporary and will sort itself out in 5 days or if all wounds are repaired.
  \end{itemize}
  \item \textbf{Repairs} Damage can be repaired similar to how a character heals, except there is no golden hour. Repairs can not be completed in space but need to be completed when docked or landed. First an engineer has to roll a repair roll modified by the ships wound level. 
  \begin{itemize}
    \item A success repairs one wound and a raise removes 2. Any more raises have no effect. Each repair attempt takes a fixed amount of days depending on spacecraft's Bulk die type plus Size modifier (this time can be reduced by 1 for every engineer who makes a success repair roll, 2 with a raise)
    \item A failed attempt only wastes the time but does not further harm to the ship and can be retried again the next day without further penalty.
    \item If no engineer is available the spacecraft can attempt to repair itself with it's own repair skill. This takes twice as long and the ship doubles the wound modifiers
  \end{itemize}
  Permanent Damage that resulted in hindrances can not be fixed at all, they are part of the ship now. Permanent damage to die types can be restored with a successful repair roll and a payment of half the difference between the current die type and the original die type (x1000) cost for spare parts. If you want to get repairs done by NPCs the ship is moored for the duration of the repairs and these take twice as long as normal. Cost for the repairs is 1\% of ships value per wound and the full difference between current die type and base die type (x10000 cost.
  \item \textbf{Maintenance:} A Spacecraft needs to be maintained or it will stop flying. Before maintenance can be completed the ship must not have any wounds left. The base maintenance cost is 0.5\% of the spacecraft's value per month. This can be lowered if the ship has an engineer that performs the maintenance himself. Each month the engineer must succeed at a repair roll modified by the ships repair mod. The maintenance itself takes the spacecraft's Bulk die type plus Size modifier in days. 
  \begin{itemize}
    \item A success halves maintenance costs. With a raise he was able to do all maintenance work without additional costs.
    \item Simple failure means the maintenance cost must be paid as calculated.
    \item On a critical failure however doubles the maintenance cost for that month
  \end{itemize}
  If maintenance is not paid on time the ship will suffer. As long as maintenance is outstanding all rolls involving the ship suffer a -1
penalty for every month of missed maintenance. In the third and every following month of neglect all attributes except Armour are reduced by one die type. This stops once the engineer succeeds in a Repair roll with a raise. If no engineer is available the accumulated maintenance has to be paid at a starport where the ship is moored for the duration that takes twice as long as normal. A ship can not maintain itself. Each lost die type must be repaired individually and takes a full maintenance cycle.
\end{itemize}

\subsection{Damage table}
\begin{redtable}{\linewidth}{ @{} L{.25} @{} L{.75} @{} }
  \textbf{Roll} & \textbf{Damage}\\
  2 & Power reduced a die type (minimum d4)\\
  3 – 4 & A random subsystem has been damaged. Gain one of the following hindrances: Brittle Armor, Corrupted Navmap, Fuel Drinker, Failing Subsystems\\
  5 – 9 & The ship has been damaged in a vital area inside the ship. Reduce by a die type one of the following: Engines, Armour, Systems\\
  10 & Wrecked Engine Capacitor; Gain the Faulty Engines hindrance. If ship already as that hindrance Engines is reduced a die type (minimum d4)\\
  11 – 12 &  Smarts reduced one die type (minimum d4)\\
\end{redtable}

\end{multicols}

\subsection{Hulls}

\begin{standardtable}{\linewidth}{ @{} L{.2} @{} L{.1} @{} L{.1} @{} L{.1} @{} L{.1} @{} L{.1} @{} L{.1} @{} L{.2} @{} L{.1} @{} }
  \textbf{Hull} & \textbf{Base Cost} & \textbf{Crew} & \textbf{Storage} & \textbf{Power} & \textbf{Size} & \textbf{Acc / TS} & \textbf{Wounds} \\
  Fighter     & 100k  & 1   & 3   & 5   & -1  & 50/700 & 1\\
  Shuttle     & 250k  & 4   & 10  & 5   & 0   & 40/500  & 2\\
  Patrol      & 250k  & 4   & 5   & 10  & 0   & 45/600  & 2\\
  Corvette    & 500k  & 10  & 20  & 15  & +1  & 35/400  & 3\\
  Frigate     & 500k  & 10  & 15  & 20  & +2  & 35/400  & 3\\
  Cruiser     & 1m    & 15  & 30  & 50  & +4  & 30/300  & 4\\
  Battleship  & 2m    & 30  & 50  & 75  & +8  & 25/200  & 4\\
  Capital     & 5m    & 45  & 100 & 100 & +10 & 20/200  & 5\\
\end{standardtable}

\begin{multicols}{2}

These are the default hull statistics for your spacecraft. After you choose a hull type, these values will be modified by your spacecraft's attributes to give your vessel a feel of it's own.

\begin{itemize}
  \item \textbf{Base Cost} is the base multiplier used to calculate the cost of your spacecraft, depending on your attributes
  \item \textbf{Crew} is the standard minimum crew required to operate the spacecraft. Each hull type includes living quarters to support their minimum crew numbers (usually in the form of bunk beds and shared living space). To increase this number you need to convert storage space into living quarters
  \item \textbf{Storage} is the amount free storage the hull can support (modified by your Bulk attribute). This is used to install fittings, or be used as cargo space
  \item \textbf{Power} is the amount of power the hull can deliver(modified by your Power attribute). This is used to power ship systems and extra fittings
  \item \textbf{Size} modifier works like the Size modifier for normal characters. Add the modifier to your spacecraft's toughness, and when attacking a spacecraft 2 or more levels smaller than you will incur a -2 attack penalty; attacking a spacecraft 4 or more levels larger than you will gain a +2 bonus; attacking a spacecraft 8 or more levels larger than you will gain a +4 bonus.
  \item \textbf{Acc / TS} field designates your base acceleration and top speed (only applicable with friction) for the hull. The Acceleration value determines how many squares per turn it can increase its speed. It can decrease its speed by twice it's Acceleration. The Top Speed value determines the upper movement limit. There are some additional considerations regarding movement:
  \begin{itemize}
    \item \textbf{Collisions:} The damage to vehicles (and its passengers) is 1d6 for every multiple of 5 of the spacecraft's speed. Increase this damage if you collide with another vehicle moving towards you. Add up the speed of both vehicles and add them together. Vehicle with dedicated armour subtract their armour from the damage. Anyone wearing safety harnesses take half damage.
    \item \textbf{Speeding:} It is easier to maneuver at slower speeds. If your current speed is over 15, apply a -2 handling penalty on the pilot. At over 30, the penalty is -4
    \item \textbf{Turning:} Vehicles make a turn at a 45 degree angle per round (there is a turning template in the books). To make a tighter turn, you must make a Maneuver. To perform a maneuver, describe to the GM what it is you exactly wish to do. The GM determines a suitable penalty (from -1 to -4), and then you can make a spacecraft maneuver roll to see if you pull it off (the pilot can use their piloting skill to cooperate).
  \end{itemize}
\end{itemize}
  
\subsection{Attributes}

\subsubsection{Armour}

Armour generally represents your hull integrity and ability to take damage. Taken with the "Size" value of your hull, it will calculate the Toughness value for your spacecraft.

\begin{standardtable}{\linewidth}{sb}
  \textbf{Die} & \textbf{Cost}\\
  d4  & x0.5\\
  d6  & x1\\
  d8  & x1.5\\
  d10 & x2\\
  d12 & x3\\
\end{standardtable}

\subsubsection{Engines}

Engines represent how fast your spacecraft can go, and gives you the ability to make a FTL jump. Normal impulse within a star system uses negligible amounts of fuel (and can be supplemented by your Power core), but FTL jumps require a whole lot of fuel to operate. 

The amount of fuel required to complete a single jump is relative to your vessel's Size modifier (negatives and zero are considered as 1) which is modified by your Engine's FTL Fuel efficiency. For example a Capital-class spaceship (+10 Size) with a d12 Engine (x0.5 FTL Fuel efficiency) so it will require 5 units of fuel to make 1 FTL Jump. Your Engine can store up to 2 jumps worth of FTL fuel internally (but extra fuel storage can be purchased if you want to go further). You can refuel in any space dock or port and on average costs \$1000 (modified by the local market).

Travel within a region in a sector (planetary body or space station and it's moons, asteroids, and immediate objects) will take on average 6 hours. Travel to another region in a sector (another planet or space station within a sector) will take on average 48 hours. Travel using FTL will take on average 6 days per hex.

Round up to nearest whole number.

\begin{standardtable}{\linewidth}{sssb}
  \textbf{Die} & \textbf{FTL Fuel} & \textbf{Cost} & \textbf{Acc / TS}\\
  d4  & x1.5 & x0.5 & x1\\
  d6  & x1 & x1 & x1\\
  d8  & x0.5 & x1.5 & x1\\
  d10 & x0.5 & x2   & x1.5\\
  d12 & x0.5 & x2.5 & x2\\
\end{standardtable}

\subsubsection{Power}

Power cores run all of the computer system on the spacecraft (with the exception of the Engine). Larger power cores allow you to operate more systems at once, but there is nothing stopping the cost-savvy captain from using a smaller power core and diverting power to systems only when they need to.

Round up to nearest whole number.

\begin{standardtable}{\linewidth}{sbb}
  \textbf{Die} & \textbf{Power} & \textbf{Cost}\\
  d4  & x0.5 & x0.5\\
  d6  & x1   & x1\\
  d8  & x1.5 & x1.5\\
  d10 & x2   & x2\\
  d12 & x3   & x2.5\\
\end{standardtable}

\subsubsection{Bulk}

Equipment and extra crew require space, and a larger spacecraft allows you to hold more. Storage space allows you to install new equipment, hold cargo, and create new quarters for extra crew. A single unit of storage is equivalent to a 2m cube and can store up to 500kg of secured goods. The Base Crew value for your hull type shows you the minimum viable crew to operate your spacecraft, and extra crew require 1 unit of Storage as living quarters.

Round up to nearest whole number.

\begin{standardtable}{\linewidth}{sbb}
  \textbf{Die} & \textbf{Storage} & \textbf{Cost}\\
  d4  & x1   & x0.5\\
  d6  & x1.5 & x1\\
  d8  & x2   & x1.5\\
  d10 & x2.5 & x2\\
  d12 & x3   & x2.5\\
\end{standardtable}

\subsubsection{Systems}

Systems govern the effectiveness of your onboard sensors, weapons, and computers. Better systems means more efficient ship operations.

Round up to nearest whole number.

\begin{standardtable}{\linewidth}{sbb}
  \textbf{Die} & \textbf{Cost}\\
  d4  & x1\\
  d6  & x1.5\\
  d8  & x2\\
  d10 & x2.5\\
  d12 & x3\\
\end{standardtable}

\subsection{Spacecraft Systems}

Every spacecraft comes with an onboard computer that handles the operation of different ship functions. While an operator is required to use these systems (and can give a bonus to the system roll), the Spacecraft uses it's own system die with wild die to determine the outcome. 

\subsubsection{Auto-pilot}

An auto-pilot is a simple Ship AI that handles spacecraft movement when you do not have a pilot. The auto-pilot automatically manages the engines, FTL drive and all the little thrusters set around the hull to give the vessel better maneuverability. The auto-pilot system is used when calculating the Evade skill when opposing enemy weapon systems by juking and diving in random patterns. You can use the Pilot's piloting skill if you do not have an auto-pilot.

\subsubsection{ECM}

ECM stands for Electronic Counter Measures. It can disrupt enemy sensors and weapon targeting systems. To successfully operate the ECM system requires an opposed Sensor roll; every success and raise means a +1 bonus to your Evade or Piloting skill.

\subsubsection{Navigation}

Navigating the Black can be a treacherous thing, and a up-to-date and accurate navigation system can be the most important system for any crew. When plotting a course for an FTL jump, make a Navigation or Astrogation (Character skill) roll. For a success you successfully plot an optimal course (each raise will shave off travel time and use more fuel). A failure means that you suffer a complication (end up at the wrong destination, use more fuel than intended, or malfunction in some way). 

Navigation systems can also be used to safely plot a course through treacherous territory, such as asteroid belts or solar flares.

\subsubsection{Operations}

Operations are a catch-all term to refer to any other type of system that does not fall into one of the other categories. This includes Medical bays, Hydroponic bays, Manufactories, and Cortical Stack back-up arrays. Some of these systems may require a dedicated skill, other when roll a Smarts check when using an operations system.

\subsubsection{Repair}

When operating in the vacuum of space it is of extreme importance to keep your spacecraft running. Repair systems allow you to automatically seal any hull breaches, diagnose systems failures, and automatically repair damaged systems.

\subsubsection{Sensors}

Sensors give your spacecraft information about environment outside the spacecraft. Depending on the complexity of your sensor you can information such as:

\begin{itemize}
  \item Atmospheric information of an undiscovered planet
  \item Make and model of a spacecraft, as well as it's weapons systems
  \item Set proximity alerts of dangerous objects such as debris, meteors and asteroids
  \item Provide a weapons lock on an enemy spacecraft (confers a +2 bonus on next round)
\end{itemize}

\subsubsection{Weapons}

Weapons are all about launching your deadly armada against the enemy. Bonuses to your attack can be gained by providing a targeting lock via your Sensors or Piloting your spacecraft so that your maneuver your enemy just where you want them. The Weapons system of your spacecraft can only operate a number of weapons equal to your Weapons die.

\end{multicols}

\subsection{Spacecraft Hindrances}

\begin{powertable}{ @{} p{.25\textwidth} @{} p{.15\textwidth} @{} p{.55\textwidth} @{} }
  \textbf{Hindrance} & \textbf{Type} & \textbf{Effect}\\
  Brittle Armor      & Major         & Your hull plating is of low quality. -1 Toughness\\
  Corrupted Navmaps  & Minor         & +d6 space travel time\\
  Defunct Scanner    & Minor         & 50\%chance on hit that scanners die, resulting in -2 to shooting and notice\\
  Failing Subsystems & Major         & -2 Repair, Roll of 1 causes Malfunction\\
  Faulty Engines     & Minor         & -1 Acc/ -2 TS\\
  Fuel Drinker       & Minor         & Double the amount of Fuel needed to make 1 jump\\
  Old Pot            & Major         & -1 to Armour and Engine, +2 system points\\
  Quirk              & Minor         & Something minor does not work correctly\\
  Toothless          & Major         & The ship type was not designed with combat in mind. In opposed rolls, enemies receive +1\\
  Unlucky            & Major         & Critical 1's on spacecraft system rolls cannot be re-rolled by spending a Bennie\\
  Wanted             & Minor/Major   & The ship is wanted by someone\\
\end{powertable}

\subsection{Spacecraft Edges}

\begin{powertable}{ @{} p{.25\textwidth} @{} p{.20\textwidth} @{} p{.5\textwidth} @{} }
  \textbf{Edge} & \textbf{Requirement} & \textbf{Effect}\\
  Advanced Auto-repair System & Seasoned, Systems d10+ & +1 to repair rolls, halve repair time with raise\\
  Afterburner & Novice, Engine d6+ & Gain Engine dice as bonus to Acceleration for d6 rounds\\
  Armor-plating & Seasoned & Toughness +1\\
  Capital destroyer & Novice, Armour d8+ & +1d8 damage when shooting at large spacecraft\\
  Caring Crew & Novice & Crew can use Bennie's on ship rolls\\
  Combat Circuitry & Seasoned, Engines d8+ & +2 to recover from Shaken\\
  Famous Weapon & Novice, Weapons d10 & +1 shooting with a specific weapon\\
  Fuel Efficient & Novice, Engines d8+ & Halve the amount of Fuel needed to make one jump\\
  Improved firing Line & Seasoned & Can use Crew Bennies on damage rolls\\
  Improved Sensor Array & Novice, Systems d8+ & +1 to notice checks\\
  Maneuver Jets  & Novice, Maneuver d6+ & +1 to piloting rolls for maneuvers\\
  Maneuver Jets (Imp.) & Maneuver Jets & +2 to piloting rolls for maneuvers\\
  Prototype Astrogation & Novice, Systems d6+ & +2 on Astronautics rolls\\
  Proximity Alert & Novice & Notice at -2 to detect surprise attackers and other danger\\
  Trusty old ship & Veteran, Systems d6+ & 2 points to spend on systems\\
  Well Built & Novice & +2 to Power rolls when Incapacitated\\
\end{powertable}

\subsection{Spacecraft Fittings}

\begin{multicols}{2}

\subsubsection{Fittings}

\begin{genericsection}{All-Terrain Landing Gear}
\textbf{Cost:} 10K per Size modifier (negatives and zero count as 1)\\
\textbf{Storage:} 2 + Size modifier\\
\textbf{Power:} 2 + Size modifier\\
The spacecraft can land on almost any terrain (without this landing gear your spacecraft can only dock with other space vessels and stations
\end{genericsection}

\begin{genericsection}{Armor}
\textbf{Cost:} 20K per Size modifier (negatives and zero count as 1)\\
\textbf{Storage:} 2 + Size modifier\\
\textbf{Power:} 0\\
Provides an additional +2 Armor
\end{genericsection}

\begin{genericsection}{Automatic targeting}
\textbf{Cost:} 100K\\
\textbf{Storage:} 1\\
\textbf{Power:} 1\\
Provides a +2 bonus to one weapon's Shooting or Weapon roll
\end{genericsection}

\begin{genericsection}{Cloaking Device}
\textbf{Cost:} 300K per Size modifier (negatives and zero count as 1)\\
\textbf{Storage:} 1 + Size modifier\\
\textbf{Power:} 1 + Size modifier\\
Requires a successful Operations (either System or Character) roll. Provides a -4 penalty on enemy Notice or Sensor rolls.
\end{genericsection}

\begin{genericsection}{Cryo Systems}
\textbf{Cost:} 10K\\
\textbf{Storage:} 2\\
\textbf{Power:} 2\\
Converts 1 unit of storage so that it is suitable for the transportation of food and people in cryostasis. (The other storage unit holds the cryo refrigeration unit). Stores up to 10 people in cryostasis, and it takes 1 hour to put and remove a person from cryostasis.
\end{genericsection}

\begin{genericsection}{ECM cluster}
\textbf{Cost:} 150K\\
\textbf{Storage:} 5\\
\textbf{Power:} 5\\
Requires a successful ECM or Operations roll. When engaged will disrupt enemy sensors for a -1 penalty to all enemy attack rolls for one round (-2 on a raise).
\end{genericsection}

\begin{genericsection}{Emergency Capsules}
\textbf{Cost:} 5K\\
\textbf{Storage:} 1\\
\textbf{Power:} -\\
Each capsule carries up to 5 people. Uses an autopilot to find a safe landing (can be overridden manually)
\end{genericsection}

\begin{genericsection}{Fuel scoops}
\textbf{Cost:} 15K\\
\textbf{Storage:} 3\\
\textbf{Power:} 1\\
Can scoop and convert fuel directly from gas giants, nebulae and other sources. Only stores 1 unit of fuel internally, but can use Fuel tanks to store additional fuel. Scooping 1 unit of fuel takes 48 hours
\end{genericsection}

\begin{genericsection}{Fuel tanks}
\textbf{Cost:} 5K\\
\textbf{Storage:} 1\\
\textbf{Power:} -\\
Adds additional fuel for one more FTL jump. Can also be adjusted to store chemical trade goods (it costs 500 to clean the storage thoroughly if you wish to change chemicals)\\
\end{genericsection}

\begin{genericsection}{Hydroponics}
\textbf{Cost:} 10K\\
\textbf{Storage:} 1\\
\textbf{Power:} 2\\
Each Hydroponics unit produces enough food and water resources for 4 people
\end{genericsection}

\begin{genericsection}{Livestock Storage}
\textbf{Cost:} 10K\\
\textbf{Storage:} 1\\
\textbf{Power:} -\\
Each unit enables the transport of livestock. Roughly estimate how many livestock can fit in 1 storage unit (2x2 cube) and multiply accordingly.
\end{genericsection}

\begin{genericsection}{Magnetic Grappler}
\textbf{Cost:} 20K\\
\textbf{Storage:} 5\\
\textbf{Power:} 2\\
Allows the grappling of another ship. Has a very short range of up to 1 square. Use an Operations (System or Character) roll against the target's Evade. If the ship is the same size then incur a -2 penalty, -4 if larger. On a success you have disabled and grappled the target. If it is a larger Size category the victim will free itself on a club on your initiative card until you have successfully boarded or otherwise fully disabled the target
\end{genericsection}

\begin{genericsection}{Med-bay}
\textbf{Cost:} 20K\\
\textbf{Storage:} 1\\
\textbf{Power:} 1\\
Fully supplied and operational medical facilities, including surgery. +2 to all medical rolls
\end{genericsection}

\begin{genericsection}{Point Defense Lasers}
\textbf{Cost:} 50K\\
\textbf{Storage:} 2\\
\textbf{Power:} 4\\
Requires a successful ECM or Operations roll against one missile or large ballistics. On a success, the target missile is denoted, or the large ballistics is destroyed
\end{genericsection}

\begin{genericsection}{Power Generator}
\textbf{Cost:} 50K\\
\textbf{Storage:} 2\\
\textbf{Power:} -\\
Increases your available power by +1\\
\end{genericsection}

\begin{genericsection}{Quarters (Amenities)}
\textbf{Cost:} 2K\\
\textbf{Storage:} 1\\
\textbf{Power:} -\\
Provides showering and toilet facilities
\end{genericsection}

\begin{genericsection}{Quarters (Living Space)}
\textbf{Cost:} 2K\\
\textbf{Storage:} 1\\
\textbf{Power:} -\\
Space for a kitchenette or entertainment units. Provides quality of life for crew
\end{genericsection}

\begin{genericsection}{Quarters (Barracks)}
\textbf{Cost:} 10K\\
\textbf{Storage:} 1\\
\textbf{Power:} -\\
Sleeps up to 4 people, with only a single locker for each. Increases your minimum Crew by +4, or can be used for passengers
\end{genericsection}

\begin{genericsection}{Quarters (Crew)}
\textbf{Cost:} 2K\\
\textbf{Storage:} 1\\
\textbf{Power:} -\\
Sleeps up to 2 people, with lockers and storage space. Increases crew contingent by +2, or can be used for passengers
\end{genericsection}

\begin{genericsection}{Quarters (Standard)}
\textbf{Cost:} 1K\\
\textbf{Storage:} 1\\
\textbf{Power:} -\\
Sleeps up to 1 person, with plenty of space to spare. Increases crew contingent by +1, or can be used for passengers
\end{genericsection}

\begin{genericsection}{Quarters (Luxury)}
\textbf{Cost:} 10K\\
\textbf{Storage:} 3\\
\textbf{Power:} -\\
Provides comfortable accommodations for 1 person, with private showering and toilet facilities. Includes additional living space with kitchenette and entertainment units.
\end{genericsection}

\begin{genericsection}{Stabilizer}
\textbf{Cost:} 75K\\
\textbf{Storage:} 2\\
\textbf{Power:} 3\\
Requires and Operations (System or Character) roll. +1 bonus on each success and raise for Piloting rolls
\end{genericsection}

\begin{genericsection}{Turbo Engine}
\textbf{Cost:} 75K\\
\textbf{Storage:} 1\\
\textbf{Power:} 3\\
Adds a 25\% bonus to Acceleration and Top Speed. Maximum of 4 can be used (for a bonus of 100\%)
\end{genericsection}

\begin{genericsection}{Workshop}
\textbf{Cost:} 5K\\
\textbf{Storage:} 1\\
\textbf{Power:} 1\\
Tech workshops for maintenance and repair. +2 to all repair rolls
\end{genericsection}

\subsubsection{Weapons}

There are three types of weapons used in spacecraft:

\begin{itemize}
  \item \textbf{Fixed} weapons require the Pilot to align the spacecraft and weapon towards the target
  \item \textbf{Turrent} weapons require a Weapon or Shooting roll to target enemy ships, but do require the target to be in it's field of view
  \item \textbf{Missile} weapons are semi-autonomous, self-adjusting weapons. They require a lock on to a target which can be either a Piloting, Shooting or Weapons roll. The target is notified of the lock and have a number of rounds depending on the range to Evade: one at short, two at medium and three at long. Evading a missile requires a Piloting or Auto-Pilot roll at -4.
\end{itemize}

A weapon can be located in 6 different places: \textbf{Rear}, \textbf{Front}, \textbf{Top}, \textbf{Bottom}, \textbf{Starboard} and \textbf{Port}.

\begin{genericsection}{Auto Cannon}
\textbf{Cost:} 100K\\
\textbf{Storage:} 2\\
\textbf{Power:} 3\\
\textbf{Type:} Fixed (double the cost for a Turrent)\\
\textbf{Damage:} 4d6+1\\
\textbf{AP:} 4\\
\textbf{Range:} 50/100/200\\
\textbf{RoF:} 3\\
\textbf{Shots:} 100\\
Can be fired using Auto rules, and is considered a Heavy Weapon. Shaves off a projectile from a metal alloy block and accelerates it to a high speed. Each block of alloy \$500
\end{genericsection}

\begin{genericsection}{Blaster}
\textbf{Cost:} 75K\\
\textbf{Storage:} 2\\
\textbf{Power:} 4\\
\textbf{Type:} Fixed (double the cost for a Turrent)\\
\textbf{Damage:} 3d6+1\\
\textbf{AP:} 3\\
\textbf{Range:} 40/80/160\\
\textbf{RoF:} 1\\
\textbf{Shots:} -\\
Is considered a Heavy Weapon. Fires a modulated and focused laser beam at a target. This weapon does not need to be reloaded as long as it is Powered. Powering up and down the weapon takes 1 round.
\end{genericsection}

\begin{genericsection}{Mass Driver}
\textbf{Cost:} 150K\\
\textbf{Storage:} 4\\
\textbf{Power:} 4\\
\textbf{Type:} Fixed (double the cost for a Turrent)\\
\textbf{Damage:} 4d8+1\\
\textbf{AP:} 5\\
\textbf{Range:} 75/150/300\\
\textbf{RoF:} 1\\
\textbf{Shots:} 200\\
Is considered a Heavy Weapon. Similar to an Auto Cannon, but fires larger projectiles at higher speeds. Each alloy blocks costs \$1000
\end{genericsection}

\begin{genericsection}{Missile Launcher (Light)}
\textbf{Cost:} 150K\\
\textbf{Storage:} 4\\
\textbf{Power:} 4\\
\textbf{Type:} Missile\\
\textbf{Damage:} 4d8\\
\textbf{AP:} 6\\
\textbf{Range:} 150/300/600\\
\textbf{RoF:} 2\\
\textbf{Shots:} Special\\
Medium Burst, Heavy Weapon. Missiles cost 500 credits each
\end{genericsection}

\begin{genericsection}{Missile Launcher (Heavy)}
\textbf{Cost:} 250K\\
\textbf{Storage:} 6\\
\textbf{Power:} 4\\
\textbf{Type:} Missile\\
\textbf{Damage:} 5d8\\
\textbf{AP:} 6\\
\textbf{Range:} 150/300/600\\
\textbf{RoF:} 2\\
\textbf{Shots:} Special\\
Medium Burst, Heavy Weapon. Missiles cost 750 credits each
\end{genericsection}

\begin{genericsection}{Missile Launcher (Armour Piercer)}
\textbf{Cost:} 350K\\
\textbf{Storage:} 6\\
\textbf{Power:} 4\\
\textbf{Type:} Missile\\
\textbf{Damage:} 5d8\\
\textbf{AP:} 150\\
\textbf{Range:} 150/300/600\\
\textbf{RoF:} 1\\
\textbf{Shots:} Special\\
Medium Burst, Heavy Weapon. Missiles cost 1500 credits each
\end{genericsection}

\begin{genericsection}{Multifocal Laser}
\textbf{Cost:} 25K\\
\textbf{Storage:} 1\\
\textbf{Power:} 1\\
\textbf{Type:} Fixed (double the cost for a Turrent)\\
\textbf{Damage:} 3d4+1\\
\textbf{AP:} 2\\
\textbf{Range:} 40/80/160\\
\textbf{RoF:} 1\\
\textbf{Shots:} -\\
Heavy Weapon. Twinned assay and penetration lasers modulate the frequency of this beam for remarkable armor penetration.
\end{genericsection}

\begin{genericsection}{Plasma Cannon}
\textbf{Cost:} 350K\\
\textbf{Storage:} 5\\
\textbf{Power:} 5\\
\textbf{Type:} Fixed (double the cost for a Turrent)\\
\textbf{Damage:} 4d10+1\\
\textbf{AP:} 7\\
\textbf{Range:} 40/80/160\\
\textbf{RoF:} 1\\
\textbf{Shots:} 100\\
Heavy Weapon. Superheats a hydrogen pellet until it reaches it’s plasma state, and then accelerates the plasma to it’s target by magnetic coils.. Ammunition costs \$1000 credits each
\end{genericsection}

\begin{genericsection}{Sandthrower}
\textbf{Cost:} 50K\\
\textbf{Storage:} 1\\
\textbf{Power:} 1\\
\textbf{Type:} Fixed\\
\textbf{Damage:} 4d4 / 3d4 / 2d4\\
\textbf{AP:} 15\\
\textbf{Range:} 30/60/120\\
\textbf{RoF:} 1\\
\textbf{Shots:} Special\\
Heavy Weapon, +2 to your Shooting or Weapons roll. Projects a spray of tiny, dense particulate matter. Sandthrowers are highly effective against lightly-armored fighters
\end{genericsection}

\end{multicols}
