\documentclass[10pt,twoside,onecolumn]{article}
\usepackage{dnd}
\usepackage{chessboard,subcaption,caption}
\usepackage{lipsum}
\usepackage[T1]{fontenc} % T1 = italiano ci aiuta per scrivere le parole italiane
\usepackage[italian]{babel} % formatta le parole italiano nel documento

\usepackage[utf8]{inputenc}
% Clickable table of content links
\usepackage[hidelinks]{hyperref}
\usepackage{bookman}                  % Closest built-in font I could find
% Start document
\begin{document}
	\tableofcontents
	\newpage
	
	% =================
	% Your content goes here
	% =================
	\section{La Scacchiera}
	Gli scacchi sono un gioco di strategia che si sviluppa su una tavola quadrata, detta "scacchiera", divisa in 64 caselle (chiamate anche case), alternativamente di colore chiaro e scuro. La scacchiera deve essere orientata in modo che la casa nell'angolo in basso a destra di ciascun giocatore sia bianca (chiara).
	
	\begin{figure}[h]
		\centering
		\chessboard[showmover=false, 
		pgfstyle=border,
		linewidth=0.01ex,
		backregions={a8-a8, h1-h1},
		labeltoplift=2mm, % padding about coordinates and chessboard side of nord
		label=true,
		hlabel=false,
		vlabel=false
		]
		\caption{}
		% \label{fig:nome_label}
	\end{figure}
	Gli scacchi sono un gioco di strategia che si sviluppa su una tavola quadrata, detta "scacchiera", divisa in 64 caselle (chiamate anche case), alternativamente di colore chiaro e scuro. La scacchiera deve essere orientata in modo che la casa nell'angolo in basso a destra di ciascun giocatore sia bianca (chiara).

\begin{figure}
	\centering
	\begin{subfigure}{.5\textwidth}
		\centering
		\chessboard[showmover=false, 
		labeltoplift=2mm, % padding about coordinates and chessboard side of nord
		label=true
		]            
		\caption{test seconda scacchiera}
	\end{subfigure}%
\end{figure}

\begin{quotebox}
	\textbf{Lo sapevi che}: Il campo da gioco della dama italiana è chiamato damiera e a differenza della scacchiera in basso a destra ha una casa nera detta cantone
\end{quotebox}

Ogni lato della scacchiera è composto da 8 case, ai bordi questa tavola troviamo delle lettere scritte in stampatello minuscolo, che vanno dalla a all’h per indicare le linee verticali (chiamate colonne) e dei numeri per indicare le linee orizzontali che vanno da 1 a 8 (chiamate traverse).
Tutte le case della scacchiera hanno un nome identificato dalla lettera che si trova al bordo della traversa e del numero che si trova in alto o in basso alla colonna, un po’ come nel gioco della battaglia navale.

La casa contrassegnata da \circledmark{*} è chiamata \textbf{e4}, non \textbf{E4} oppure \textbf{4e}.


\subsection{Elements}
\subsubsection{Boxes}

\begin{commentbox}{Neat Green Box!}
    \lipsum[2]
\end{commentbox}

\begin{quotebox}
    As you approach this template you get a sense that the blood and tears of many generations went into its making. A warm feeling welcomes you as you type your first words.
\end{quotebox}

\begin{paperbox}{Do the Players need direction?}
    \lipsum[2]
\end{paperbox}


\subsubsection{Monster}

\begin{monsterbox}{Monster Foo}
    \textit{Small metasyntatic variable (golbinoid), neutral evil}\\
    \hline
    \basics[%
    armorclass = 12,
    hitpoints  = 16 (3d8 + 3),
    speed      = 50 ft
    ]
    \hline
    \stats[
    STR = 12 (+1),
    DEX = 14 (+2)
    ]
    \hline
    \details[%
    % If you want to use commas in these sections, enclose the
    % description in braces.
    % I'm so sorry.
    languages = {Common Lisp, Erlang},
    ]
    \hline \\[1mm]
    \begin{monsteraction}[Monster-super-powers]
        This Monster has some serious superpowers!
    \end{monsteraction}
    \monstersection{Actions}
    \begin{monsteraction}[Generate text]
        This one can generate tremendous amounts of text! Though only when it wants to.
    \end{monsteraction}

    \begin{monsteraction}[More actions]
    See, here he goes again! Yet more text.
    \end{monsteraction}
\end{monsterbox}

\subsubsection{Spell}

\begin{spellbox}{Spell Generic}
    \spelldetails[]
    \begin{spellaction}[Materials]
    This spell doesn't really require much.
    \end{spellaction}

    \begin{spellaction}[Effect]
    Touch things and get dissaproving looks.
    \end{spellaction}
\end{spellbox}

% End document
\end{document}
