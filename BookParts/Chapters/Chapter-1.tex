	Gli scacchi sono un gioco di strategia che si sviluppa su una tavola quadrata, detta "scacchiera", divisa in 64 caselle (chiamate anche case), alternativamente di colore chiaro e scuro. La scacchiera deve essere orientata in modo che la casa nell'angolo in basso a destra di ciascun giocatore sia bianca (chiara).

\begin{figure}
	\centering
	\begin{subfigure}{.5\textwidth}
		\centering
		\chessboard[showmover=false, 
		labeltoplift=2mm, % padding about coordinates and chessboard side of nord
		label=true
		]            
		\caption{test seconda scacchiera}
	\end{subfigure}%
\end{figure}

\begin{quotebox}
	\textbf{Lo sapevi che}: Il campo da gioco della dama italiana è chiamato damiera e a differenza della scacchiera in basso a destra ha una casa nera detta cantone
\end{quotebox}

Ogni lato della scacchiera è composto da 8 case, ai bordi questa tavola troviamo delle lettere scritte in stampatello minuscolo, che vanno dalla a all’h per indicare le linee verticali (chiamate colonne) e dei numeri per indicare le linee orizzontali che vanno da 1 a 8 (chiamate traverse).
Tutte le case della scacchiera hanno un nome identificato dalla lettera che si trova al bordo della traversa e del numero che si trova in alto o in basso alla colonna, un po’ come nel gioco della battaglia navale.

La casa contrassegnata da \circledmark{*} è chiamata \textbf{e4}, non \textbf{E4} oppure \textbf{4e}.
